\documentclass[a4paper,12pt]{article}
\usepackage{geometry}
\geometry{margin=1in}
\usepackage{amsmath, amssymb}
\usepackage{enumitem}
\usepackage{graphics}

\title{\textbf{CCCC Interviews 2024}}
\author{Coding Club Competitive Coding}
\date{January 31, 2025}

\begin{document}

\maketitle
\hrulefill

\section*{Instructions}
\begin{itemize}
    \item There are 6 questions in this paper, some of which have multiple subparts
    \item After solving a question/subpart, you must explain your solution to one of the CCCC members listed at the end of the question
    \item The number of points allotted to each subpart is inversely proportional to the number of people who solve it
    \item Duration: 3 hours.
\end{itemize}

\hrulefill

\section*{Question 1}
Let's start with a famous problem:

\begin{enumerate}[label=\alph*)]
    \item Given an array $A$ of integers of size $N$. Count the number of subarrays that sum to $0$.Can you generalize it for any given target sum $K$?

\item You are given an array $a_1, a_2, \dots, a_n$ consisting of integers from $0$ to $9$. A subarray $a_l, a_{l+1}, a_{l+2}, \dots, a_{r-1}, a_r$ is good if the sum of elements of this subarray is equal to the length of this subarray
    \begin {center}
    $\sum_{i=l}^{r}a_i = r - l + 1$.
    \end {center}
    For example, if $a = [1, 2, 0]$, then there are $3$ good subarrays: $a_{1 \dots 1} = [1], a_{2 \dots 3} = [2, 0], a_{1 \dots 3}=[1, 2, 0]$. Calculate the number of good subarrays of the array $a$.

\item You are given an integer array $a_1, a_2, \dots, a_n (1 \le a_i \le n)$. Find the number of subarrays of $a$ whose $XOR$ has an even number of divisors. In other words, find all pairs of indices $(i \le j)$ such that $a_i \oplus a_{i+1} \oplus \dots \oplus a_j$ has an even number of divisors.

For example, numbers 2, 3, 5 or 6 have an even number of divisors, while 1 and 4 — odd. Consider that 0 has an odd number of divisors in this task.
\end{enumerate}

\textbf{[Answer to Vansh/Ashish]}

\hrulefill

\section*{Question 2}

A group of $N$ people are on a hiking trip. It starts raining heavily and the group starts looking for places where they can be protected from the rain. There are $2$ enclosures nearby where people can find shelter. Let’s call them $A$ and B. Enclosure $A$ can accommodate $X$ people, and Enclosure $B$ can accommodate $Y$ people. It is guaranteed that $(X + Y) \ge N$, and so everyone can be accommodated. The people are numbered from $1$ to $N$ . We know how far each person is from each enclosure. More specifically, for person $i$, $A_{dist}[i]$ denotes how far the $ith$ person is from $A$ and $B_{dist}[i]$ denotes how far the $ith$ person is from $B$. The group has genuine camaraderie, so they decide that they want to minimize the total distance traveled by their group members altogether to reach their respective enclosures. Given $N , X, Y$ and the distances $(A_{dist}[i], B_{dist}[i]), 1 \le i \le N$, your aim is to compute this minimum value.

For example, if $N = 4, X = 2, Y = 2$, and the distances are:
\begin{center}
    $A_{dist} = [10, 23, 15, 5]$

    $B_{dist} = [12, 20, 8, 20]$
\end{center}
Then the optimal strategy would be for $1$ and $4$ to go to $A$ and for $2$ and $3$ to go to $B$. In this case the total distance traveled is $10 + 5 + 20 + 8 = 43$

\textbf{[Answer to Siddhant/Ashish]}

\hrulefill

\section*{Question 3}
 Consider a grid representation of an art gallery, where the layout is defined by a non-increasing sequence $A = [A_1, A_2, \dots, A_N]$. The $ith$ row from the top contains $A_i$ tiles, aligned to the left. 
 \begin{enumerate}[label=\roman*.]
    \item A camera can be placed on any tile.
    \item A camera can face either downward or to the right.
    \item A downward-facing camera monitors the tile it is placed on and all tiles below it in the same column.
    \item A right-facing camera monitors the tile it is placed on and all tiles to the right in the same row.
    \item Each tile must be monitored by at least one camera.
    \item No camera should monitor another camera.
    \item Each tile can contain at most one camera.
\end{enumerate}

\noindent
Determine the number of valid placements of cameras modulo $10^9 + 7$.
\begin{enumerate}[label=\alph*)]
    \item Solve this in time complexity $O(N \cdot max(A_i))$
    \item Solve this in time complexity $O(N + max(A_i))$
\end{enumerate}
For clarification on time complexity and big $O$ notation, contact an invigilator.

\begin{figure}
    \begin{center}
    \includegraphics{image.jpg}

    Example of art gallery. where $A = [4, 3, 1]$
    \end{center}
\end{figure}

\textbf{[Answer to Pramit/Abheek]}

\hrulefill

\section*{Question 4}
There are $N$ safes and $N$ keys. Each key can open only one safe, and each safe can be opened by only one key. We randomly place one key into each safe. $N - M$ safes are then randomly chosen, and then locked. What is the probability that we can open all the safes with the $M$ keys in the $M$ remaining safes? Note: Once a safe is opened, the key inside the safe can be used to open another safe.

\noindent
Sub parts:
\begin{enumerate}[label=\alph*)]
    \item Solve this for the case where $M = 2 \le N$
    \item Solve this for the general case where $M \le N$ is any integer
\end{enumerate}


\textbf{[Answer to Kanav/Siddhant]}

\hrulefill

\section*{Question 5}
You have a number that is formed by concatenating the decimal representations of the first $N$ integers. This number is given to the digit adder machine. The digit adder machines takes a number $X$ as input and outputs the sum of its digits. However today the machine is stuck and will keep feeding the output back into the input until there is only 1 digit left. Find this digit for a given $N$.

For example, if $N = 12$, we have $X = 123456789101112$. This number after passing into the digit adder machine first becomes 78, 15 and then finally 6.

\noindent
Sub parts:
\begin{enumerate}[label=\alph*)]
    \item Solve this in time complexity $O(log^2N)$
    \item Solve this in time complexity $O(1)$
\end{enumerate}
For clarification on time complexity and big $O$ notation, contact an invigilator.

\textbf{[Answer to Vansh/Pramit]}

\hrulefill

\section*{Question 6}
\begin{enumerate}[label=\alph*)]
    \item There is a long hallway which is 2 feet wide and $N$ feet long. You have to tile the entire hallway using $1 \times 2$ and $2 \times 1$ tiles. Find the number of ways of tiling the entire grid such that the entire hallway is tiles and no two tiles are overlapping.
    \item The tile shop you buy tiles from has now introduced a new tile of $1 \times 1$ feet$^2$. Find the new number of ways of tiling the hallway.
    \item You have a new hallway which is 3 feet wide and $M$ feet long. Find the number of ways of tiling it using only $1 \times 2$ and $2 \times 1$ tiles
\end{enumerate}

\textbf{[Answer to Abheek/Kanav]}

\hrulefill

\begin{center}
    Best of Luck
\end{center}

\end{document}
