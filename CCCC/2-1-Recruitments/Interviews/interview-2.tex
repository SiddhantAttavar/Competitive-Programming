\documentclass[a4paper,12pt]{article}
\usepackage{geometry}
\geometry{margin=1in}
\usepackage{amsmath, amssymb}
\usepackage{enumitem}
\usepackage{graphics}
\usepackage{float}

\title{\textbf{CCCC Interviews 2024}}
\author{Coding Club Competitive Coding}
\date{February 3, 2025}

\begin{document}

\maketitle
\hrulefill

\section*{Instructions}
\begin{itemize}
    \item There are 6 questions in this paper, some of which have multiple subparts
    \item After solving a question/subpart, you must explain your solution to one of the CCCC members listed at the end of the question
    \item The number of points allotted to each subpart is inversely proportional to the number of people who solve it
    \item Duration: 3 hours.
\end{itemize}

\hrulefill

\section*{Question 1}
Consider a \textbf{XOR} pyramid where each number is the \textbf{XOR} of lower-left and lower-right numbers. Here is an example pyramid:

\begin{figure}[H]
    \begin{center}
    \includegraphics{pyramid.png}
    \end{center}
\end{figure}

Given the bottom row of the pyramid, your task is to find:
\begin{enumerate}[label=\alph*)]
    \item The topmost number of the pyramid
    \item Any arbitrary number specified by coordinates $(i, j)$ which represent the $j_{th}$ number on the $i_{th}$ row
    \item How to calculate the values if \textbf{OR} or \textbf{AND} were used instead of \textbf{XOR}
\end{enumerate}

\textbf{[Answer to Kanav/Abheek]}

\hrulefill

\section*{Question 2}

You are in charge of a counting shipping containers at a wharf. This wharf contains $N$ stacks of containers all of which are initially empty. A total of $K$ ships come to the wharf to unload their containers. The $i_{th}$ $(1 \le i \le K)$ ship unloads $l_i$ shipping containers. These shipping containers are picked up by a crane and placed in $l_i$ different stacks. In particular, for the $i_{th}$ ship, the first container is placed at position $a_i$ and every subsequent container is placed $d_i$ stacks to the right of the previous, i.e. at positions $a_i, a_i + d_i, \dots, a_i + (l_i - 1)d_i$. It is guaranteed that all $l_i$ containers can be placed in the $n$ stacks $(a_i + (l_i - 1)d_i \le N)$. Find the number of shipping containers in each of the $N$ stacks after all $K$ ships have been unloaded.

\noindent
Sub parts:
\begin{enumerate}[label=\alph*)]
    \item Solve this in time complexity $O(NK)$
    \item Solve this in time complexity $O(N + K)$ if it given that $d_1 = d_2 = \dots = d_k$
    \item Solve this in time complexity $O((N + K) \cdot \sqrt{N})$
\end{enumerate}
For clarification on time complexity and big $O$ notation, contact an invigilator.

\textbf{[Answer to Ashish/Siddhant]}

\hrulefill

\section*{Question 3}
$N$ people came to a birthday party. Then those, who had no friends among people at the party, left. Then those, who had exactly 1 friend among those who stayed, left as well. Then those, who had exactly $2, 3, \dots, N - 1$ friends among those who stayed by the moment of their leaving, did the same.

Devise an algorithm to calculate the maximum possible people that could be left in the end.

\textbf{[Answer to Vansh/Siddhant]}

\hrulefill

\section*{Question 4}

\begin{enumerate}[label=\alph*)]
    \item You are climbing a staircase with $N$ steps. At each step, you can either take 1 step or 2 steps at a time. Find the number of distinct ways to reach the top.

    \item Now, you are given an additional option: You can also take a jump of exactly 3 steps at a time. However the jump of exactly 3 steps can only be taken if the previous jump wasn't also a 3 step jump Find the new number of ways to reach the top.

    \item Consider a more general case where you have a staircase with $N$ steps, and at each step, you can take either 1, 2, or any of the first $K$ natural number steps. Find the number of distinct ways to reach the top. Note: there is no restriction on when you can take a certain like in the part $(b)$
\end{enumerate}
For all subtasks you are expected to find an algorithm which has a time complexity of $O(N)$. 
For clarification on time complexity and big $O$ notation, contact an invigilator.

\textbf{[Answer to Abheek/Vansh]}

\hrulefill

\section*{Question 5}

There are $N$ slimes standing on a number line. The $i_{th}$ slime from the left is at position $X_i$. It is guaranteed that the positions of the slimes for a strictly increasing sequences, i.e. $1 \le X_1 < X_2 < \dots < X_N$.

Gautam will perform $N - 1$ operations. The $i_{th}$ operation consists of the following procedures:
\begin{itemize}
    \item Choose an integer $k$ between 1 and $N - i$ (inclusive) with equal probability.
    \item Move the $k_{th}$ slime from the left, to the position of the neighboring slime to the right.
    \item Fuse the two slimes at the same position into one slime. 
\end{itemize}

Find the total distance traveled by the slimes multiplied by $(N - 1)!$.  If a slime is born by a fuse and that slime moves, we count it as just one slime.

\noindent
Sub parts:
\begin{enumerate}[label=\alph*)]
    \item Solve this in time complexity $O(N^2)$
    \item Solve this in time complexity $O(N)$
\end{enumerate}
For clarification on time complexity and big $O$ notation, contact an invigilator.

\textbf{[Answer to Gautam/Kanav]}

\hrulefill

\section*{Question 6}

You are given $n$ numbers $a_1, a_2, \dots, a_n$. Find a value of $x$ that minimizes the sum
\begin{center}
    $\sum_1^n {|a_i - x|^c} = |a_1 - x|^c + |a_2 - x|^c + \dots + |a_n - x|^c$
\end{center}

Sub parts:
\begin{enumerate}[label=\alph*)]
    \item Solve this for the case where $c = 1$
    \item Solve this for the case where $c = 2$
    \item Solve this for the general case where $c \in \mathbb{N}$
\end{enumerate}

\textbf{[Answer to Gautam/Nikhil]}

\hrulefill

\begin{center}
    Best of Luck
\end{center}

\end{document}
